\documentclass[11pt]{article}

\usepackage[letterpaper, margin=1in]{geometry}
\usepackage{xcolor}
\usepackage{amsmath}
\usepackage{fancyhdr}
\usepackage{multicol}
\usepackage{graphicx}
\usepackage{enumitem}
\usepackage{listings}
\usepackage{tcolorbox}
\usepackage{titlesec}
\usepackage{booktabs}
\usepackage{float}

% Define custom colors
\definecolor{codebg}{RGB}{245,245,245}
\definecolor{codeframe}{RGB}{200,200,200}
\definecolor{commentcolor}{RGB}{100,100,100}
\definecolor{keywordcolor}{RGB}{0,0,128}
\definecolor{sectioncolor}{RGB}{0,51,102}

% Customize section titles
\titleformat{\section}
  {\Large\bfseries\color{sectioncolor}}{\thesection}{1em}{}[\titlerule]
\titleformat{\subsection}
  {\large\bfseries\color{sectioncolor}}{\thesubsection}{1em}{}
\titleformat{\subsubsection}
  {\normalsize\bfseries\color{sectioncolor}}{\thesubsubsection}{1em}{}

% Header and footer
\pagestyle{fancy}
\fancyhf{}
\fancyhead[L]{\small\textit{COMP 6231 Assignment 3 - Task 1}}
\fancyhead[R]{\small\textit{Fall 2025}}
\fancyfoot[C]{\thepage}
\renewcommand{\headrulewidth}{0.4pt}
\renewcommand{\footrulewidth}{0.4pt}

% Listing style
\lstset{
    basicstyle=\ttfamily\footnotesize,
    backgroundcolor=\color{codebg},
    breaklines=true,
    frame=single,
    frameround=tttt,
    rulecolor=\color{codeframe},
    language=bash,
    breakatwhitespace=false,
    showstringspaces=false,
    aboveskip=8pt,
    belowskip=8pt,
    xleftmargin=10pt,
    xrightmargin=10pt,
    framexleftmargin=5pt,
    framexrightmargin=5pt,
    numbers=none,
    commentstyle=\color{commentcolor},
    keywordstyle=\color{keywordcolor}\bfseries
}

\title{\textbf{\Large COMP 6231 Assignment 3}\\[0.5em]
\textbf{\large Task 1: File Server Commands Explanation}}
\author{
    \small
    \begin{tabular}{c}
        Qiang Xue (40300671) \quad Yicheng Cai (26396283) \quad Yikai Chen (40302669)\\[0.3em]
        Yifan Wu (40153584) \quad Alexander Sutherland (40321783)
    \end{tabular}
}
% \date{\small Fall 2025}

\begin{document}
\maketitle
\thispagestyle{fancy}

\section{Overview}

The file server architecture utilizes Docker Swarm for container orchestration, enabling seamless communication between a central server and multiple clients across different physical machines. Each client maintains its own isolated file structure on the server, demonstrating a multi-tenant file storage system.

\section{Server Configuration}

\subsection{Docker Image Build}

\vspace{0.3em}
\noindent\textbf{Dockerfile Content:}
\begin{lstlisting}
FROM ozxx33/fileserver-base
WORKDIR /usr/src/app
COPY ./server.py .
CMD ["python", "./server.py"]
\end{lstlisting}

\vspace{0.5em}
\noindent\textbf{Build Command:}
\begin{lstlisting}
docker build -t task1_server .
\end{lstlisting}

\noindent\textit{Explanation:} This command creates a Docker image named \texttt{task1\_server}. The \texttt{-t} flag tags the image with the specified name, while the \texttt{.} indicates the build context is the current directory containing the Dockerfile and server.py file.

\subsection{Docker Swarm Initialization}

\vspace{0.3em}
\noindent\textbf{Initialize Swarm:}
\begin{lstlisting}
docker swarm init
\end{lstlisting}

\noindent\textit{Explanation:} This command initializes a Docker Swarm on the manager node. Upon initialization, the system returns a join token that worker nodes can use to join the swarm.

\vspace{0.5em}
\noindent\textbf{Retrieve Worker Token:}
\begin{lstlisting}
docker swarm join-token worker
\end{lstlisting}

\noindent\textit{Explanation:} Displays the join token for worker nodes. This token is required for worker nodes to authenticate and join the Docker Swarm cluster.
\subsection{Create Overlay Network}

\vspace{0.3em}
\noindent\textbf{Create Network:}
\begin{lstlisting}
docker network create -d overlay 6231-net
\end{lstlisting}

\noindent\textit{Explanation:} Creates an overlay network named \texttt{6231-net}. The \texttt{-d overlay} flag specifies the network driver type. 

\subsection{Verify Configuration}

\vspace{0.3em}
\noindent\textbf{List Cluster Nodes:}
\begin{lstlisting}
docker node ls
\end{lstlisting}

\noindent\textit{Explanation:} Lists all nodes in the Docker Swarm, displaying their status, availability, and role (manager or worker). This verification step ensures all nodes have successfully joined the cluster.

\vspace{0.3em}
\noindent\textbf{List Networks:}
\begin{lstlisting}
docker network ls
\end{lstlisting}

\noindent\textit{Explanation:} Lists all networks in the Docker Swarm, confirming the creation of the overlay network \texttt{6231-net}.

\vspace{0.3em}
\noindent\textbf{Inspect Specific Network:}
\begin{lstlisting}
docker network inspect 6231-net
\end{lstlisting}

\noindent\textit{Explanation:} Provides detailed information about the overlay network, including connected containers and their IP addresses.

\subsection{Run Server Container}

\vspace{0.3em}
\noindent\textbf{Launch Server Container:}
\begin{lstlisting}
docker run -it --rm --name server --hostname server 
    --network 6231-net task1_server bash
\end{lstlisting}

\noindent\textit{Explanation - Command Parameters:} 
\begin{itemize}[leftmargin=*,itemsep=2pt]
    \item \texttt{-it}: Runs container in interactive mode with terminal access
    \item \texttt{--rm}: Automatically removes the container upon termination
    \item \texttt{--name server}: Assigns the container name "server"
    \item \texttt{--hostname server}: Sets hostname for name resolution within the network
    \item \texttt{--network 6231-net}: Connects container to the overlay network
    \item \texttt{bash}: Opens bash shell for manual command execution
\end{itemize}

\vspace{0.5em}
\noindent\textbf{Verify Container IP:}
\begin{lstlisting}
hostname -i
\end{lstlisting}

\noindent\textit{Output:} \texttt{10.0.1.28}

\noindent\textit{Explanation:} Displays the IP address assigned to the container within the overlay network, enabling communication with other containers.

\vspace{0.5em}
\noindent\textbf{Execute Server Application:}
\begin{lstlisting}
python server.py
\end{lstlisting}

\noindent\textit{Explanation:} Launches the file server application, which listens on \texttt{0.0.0.0:65432} for incoming client connections and organizes received files in client-specific directories.

\section{Client Configuration}

\subsection{Docker Image Build}

\vspace{0.3em}
\noindent\textbf{Dockerfile Content:}
\begin{lstlisting}
FROM ozxx33/fileserver-base
WORKDIR /usr/src/app
COPY . .
ENV SERVER_IP=server
ENV SERVER_PORT=65432
CMD ["python", "./client.py"]
\end{lstlisting}

\vspace{0.5em}
\noindent\textbf{Build Command:}
\begin{lstlisting}
docker build -t task1_client .
\end{lstlisting}

\noindent\textit{Explanation:} Creates a Docker image named \texttt{task1\_client}. The Dockerfile copies all files from the current directory (including client.py and test files) and sets environment variables (\texttt{SERVER\_IP} and \texttt{SERVER\_PORT}) for server connection configuration.

\subsection{Distribute Client Image}

There are two methods to distribute the client image to worker nodes:

\vspace{0.3em}
\noindent\textbf{Method 1 - Image Export/Import:}
\begin{lstlisting}
# On manager node
docker save -o task1_client.tar task1_client

# On worker node
docker load -i task1_client.tar
\end{lstlisting}

\noindent\textit{Explanation:} The \texttt{docker save} command exports the image to a tar archive file, which can be transferred to worker nodes and loaded using \texttt{docker load}. This method is ideal when nodes lack access to a shared registry.

\vspace{0.5em}
\noindent\textbf{Method 2 - Build on Each Node:}

\noindent Copy the Dockerfile and necessary files to each worker node and execute the build command locally on each node.

\subsection{Join Swarm from Worker Nodes}

\vspace{0.3em}
\noindent\textbf{Join Worker to Swarm:}
\begin{lstlisting}
docker swarm join --token <token> <serverip>:2377
\end{lstlisting}

\noindent\textit{Explanation:} Joins a worker node to the Docker Swarm. The token authenticates the node, while the IP address (<serverip>) and port 2377 specify the manager node's address. Port 2377 is the default Docker Swarm management port.

\subsection{Deploy Client Containers}

\subsubsection{Client 1 (First Worker Node)}

\vspace{0.3em}
\noindent\textbf{Launch Client Container:}
\begin{lstlisting}
docker run -it --rm --name client1 --hostname client1 
    --network 6231-net task1_client bash
\end{lstlisting}

\noindent\textit{Container IP:} \texttt{10.0.1.30}

\vspace{0.5em}
\noindent\textbf{Execute Client Application:}
\begin{lstlisting}
python client.py
\end{lstlisting}

\noindent\textit{Explanation:} Initiates the client application, which connects to the server using hostname resolution via Docker's DNS and transfers files. 

\subsubsection{Client 2 (Second Worker Node)}

\vspace{0.3em}
\noindent\textbf{Launch Client Container:}
\begin{lstlisting}
docker run -it --rm --name client2 --hostname client2 
    --network 6231-net task1_client bash
\end{lstlisting}

\noindent\textit{Container IP:} \texttt{10.0.1.32}

\vspace{0.5em}
\noindent\textbf{Execute Client Application:}
\begin{lstlisting}
python client.py
\end{lstlisting}

\noindent\textit{Explanation:} Operates on a separate worker node with unique hostname.

\subsubsection{Client 3 (Third Worker Node)}

\vspace{0.3em}
\noindent\textbf{Launch Client Container:}
\begin{lstlisting}
docker run -it --rm --name client3 --hostname client3 
    --network 6231-net task1_client bash
\end{lstlisting}

\noindent\textit{Container IP:} \texttt{10.0.1.34}

\vspace{0.5em}
\noindent\textbf{Execute Client Application:}
\begin{lstlisting}
python client.py
\end{lstlisting}

\noindent\textit{Explanation:} Completes the three-client architecture.

\section{Network and Storage Configuration}

\subsection{Network Configuration}

\begin{table}[h]
\centering
\begin{tabular}{llll}
\toprule
\textbf{Container} & \textbf{Hostname} & \textbf{IP Address} & \textbf{Port} \\
\midrule
Server & server & 10.0.1.28 & 65432 \\
Client 1 & client1 & 10.0.1.30 & Dynamic \\
Client 2 & client2 & 10.0.1.32 & Dynamic \\
Client 3 & client3 & 10.0.1.34 & Dynamic \\
\bottomrule
\end{tabular}
\caption{Container network configuration within the overlay network}
\label{tab:network}
\end{table}

\subsection{Storage Configuration}

\begin{itemize}[leftmargin=*,itemsep=4pt]
    \item \textbf{Server Storage:} Files are stored within the server container at \texttt{/usr/src/app/}
    \item \textbf{Client-Specific Directories:} Each client can use \texttt{mkdir xxx} to create its own directory on the server for file storage.)
    \item \textbf{No Shared Volumes:} Each container maintains isolated storage; file transfer occurs exclusively via network communication
\end{itemize}

\section{Implementation Results}

\subsection{Server-Side View}

\begin{figure}[H]
\centering
\fbox{\includegraphics[width=0.85\textwidth]{../server/image.png}}
\caption{Server container showing received files organized by client directories}
\label{fig:server}
\end{figure}

\subsection{Client-Side Views}

\begin{figure}[H]
\centering
\fbox{\includegraphics[width=0.85\textwidth]{../client/client1.png}}
\caption{Client 1 successfully creating \texttt{client1} directory and sending files to server}
\label{fig:client1}
\end{figure}

\begin{figure}[H]
\centering
\fbox{\includegraphics[width=0.85\textwidth]{../client/client2.png}}
\caption{Client 2 successfully creating \texttt{client2} directory and sending files to server}
\label{fig:client2}
\end{figure}

\begin{figure}[H]
\centering
\fbox{\includegraphics[width=0.85\textwidth]{../client/client3.png}}
\caption{Client 3 successfully creating \texttt{client3} directory and sending \texttt{about.txt} files to server and execute \texttt{wordcount about.txt} command}
\label{fig:client3}
\end{figure}

\end{document}