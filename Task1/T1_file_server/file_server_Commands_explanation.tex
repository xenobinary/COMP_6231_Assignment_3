\documentclass{article}

\usepackage[letterpaper, margin=0.9in]{geometry}
\usepackage{xcolor}
\usepackage{amsmath}
\usepackage{fancyhdr}
\usepackage{multicol}
\usepackage{graphicx}
\usepackage{enumitem}

\pagestyle{fancy}
\fancyhf{}
\fancyhead[L]{COMP 6231 Assignment 3}
\fancyhead[R]{Fall 2025}
\fancyfoot[C]{\thepage}

\title{COMP 6231 Assignment 3}
\author{Qiang Xue - 40300671, Yicheng Cai, Yikai Chen, Yifan Wu, Alexander Sutherland}

\begin{document}
\maketitle
\section*{Task 1: }
\subsection*{File Server Commands Explanation}
\begin{enumerate}
    \item \textbf{Docker image creation:} 
    \begin{itemize}
        \item \textbf{Client:} Navigate to the \texttt{Task1/client} directory and build the Docker image for the client using the command:
        \begin{verbatim}
        docker build --tag task1_client .
        \end{verbatim}
        This command creates a Docker image named \texttt{task1\_client} based on the instructions in the \texttt{Dockerfile} located in the current directory. The content of the \texttt{Dockerfile} is as follows:
        \begin{verbatim}
        FROM ozxx33/fileserver-base
        WORKDIR /usr/src/app
        COPY . .
        CMD ["python", "./client.py"]
        \end{verbatim}
        \item \textbf{Server:} Navigate to the \texttt{Task1/server} directory and build the Docker image for the server using the command:
        \begin{verbatim}
        docker build --tag task1_server .
        \end{verbatim}
        This command creates a Docker image named \texttt{task1\_server} based on the instructions in the \texttt{Dockerfile} located in the current directory. The content of the \texttt{Dockerfile} is as follows:
        \begin{verbatim}
        FROM ozxx33/fileserver-base
        WORKDIR /usr/src/app
        COPY ./server.py .
        EXPOSE 65432
        CMD ["python", "./server.py"]
        \end{verbatim}
    \end{itemize}
    
    \item \textbf{Running Docker containers:}
    \begin{itemize}
        \item \textbf{Server:} To run the server container, use the command:
        \begin{verbatim}
        docker run -it --name socket_server --network host task1_server
        \end{verbatim}
        This command starts a new container named \texttt{socket\_server} in interactive mode from the \texttt{task1\_server} image. The \texttt{--network host} option allows the container to share the host's network stack, enabling it to listen for incoming connections on the host's network interfaces. The \texttt{-it} option allows for interactive terminal access to the server container.
        \item \textbf{Client:} To run the client container, use the command:
        \begin{verbatim}
        docker run -it --name socket_client --network host task1_client
        \end{verbatim}
        This command starts a new container named \texttt{socket\_client} from the \texttt{task1\_client} image. The \texttt{--network host} option allows the client container to share the host's network stack, enabling it to communicate with the server running on the host machine. The \texttt{-it} option allows for interactive terminal access to the client container.
    \end{itemize} 
\end{enumerate}

\end{document}